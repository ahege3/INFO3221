% Options for packages loaded elsewhere
\PassOptionsToPackage{unicode}{hyperref}
\PassOptionsToPackage{hyphens}{url}
%
\documentclass[
]{article}
\title{HW(1)}
\author{Alex Hege}
\date{1/31/2022}

\usepackage{amsmath,amssymb}
\usepackage{lmodern}
\usepackage{iftex}
\ifPDFTeX
  \usepackage[T1]{fontenc}
  \usepackage[utf8]{inputenc}
  \usepackage{textcomp} % provide euro and other symbols
\else % if luatex or xetex
  \usepackage{unicode-math}
  \defaultfontfeatures{Scale=MatchLowercase}
  \defaultfontfeatures[\rmfamily]{Ligatures=TeX,Scale=1}
\fi
% Use upquote if available, for straight quotes in verbatim environments
\IfFileExists{upquote.sty}{\usepackage{upquote}}{}
\IfFileExists{microtype.sty}{% use microtype if available
  \usepackage[]{microtype}
  \UseMicrotypeSet[protrusion]{basicmath} % disable protrusion for tt fonts
}{}
\makeatletter
\@ifundefined{KOMAClassName}{% if non-KOMA class
  \IfFileExists{parskip.sty}{%
    \usepackage{parskip}
  }{% else
    \setlength{\parindent}{0pt}
    \setlength{\parskip}{6pt plus 2pt minus 1pt}}
}{% if KOMA class
  \KOMAoptions{parskip=half}}
\makeatother
\usepackage{xcolor}
\IfFileExists{xurl.sty}{\usepackage{xurl}}{} % add URL line breaks if available
\IfFileExists{bookmark.sty}{\usepackage{bookmark}}{\usepackage{hyperref}}
\hypersetup{
  pdftitle={HW(1)},
  pdfauthor={Alex Hege},
  hidelinks,
  pdfcreator={LaTeX via pandoc}}
\urlstyle{same} % disable monospaced font for URLs
\usepackage[margin=1in]{geometry}
\usepackage{color}
\usepackage{fancyvrb}
\newcommand{\VerbBar}{|}
\newcommand{\VERB}{\Verb[commandchars=\\\{\}]}
\DefineVerbatimEnvironment{Highlighting}{Verbatim}{commandchars=\\\{\}}
% Add ',fontsize=\small' for more characters per line
\usepackage{framed}
\definecolor{shadecolor}{RGB}{248,248,248}
\newenvironment{Shaded}{\begin{snugshade}}{\end{snugshade}}
\newcommand{\AlertTok}[1]{\textcolor[rgb]{0.94,0.16,0.16}{#1}}
\newcommand{\AnnotationTok}[1]{\textcolor[rgb]{0.56,0.35,0.01}{\textbf{\textit{#1}}}}
\newcommand{\AttributeTok}[1]{\textcolor[rgb]{0.77,0.63,0.00}{#1}}
\newcommand{\BaseNTok}[1]{\textcolor[rgb]{0.00,0.00,0.81}{#1}}
\newcommand{\BuiltInTok}[1]{#1}
\newcommand{\CharTok}[1]{\textcolor[rgb]{0.31,0.60,0.02}{#1}}
\newcommand{\CommentTok}[1]{\textcolor[rgb]{0.56,0.35,0.01}{\textit{#1}}}
\newcommand{\CommentVarTok}[1]{\textcolor[rgb]{0.56,0.35,0.01}{\textbf{\textit{#1}}}}
\newcommand{\ConstantTok}[1]{\textcolor[rgb]{0.00,0.00,0.00}{#1}}
\newcommand{\ControlFlowTok}[1]{\textcolor[rgb]{0.13,0.29,0.53}{\textbf{#1}}}
\newcommand{\DataTypeTok}[1]{\textcolor[rgb]{0.13,0.29,0.53}{#1}}
\newcommand{\DecValTok}[1]{\textcolor[rgb]{0.00,0.00,0.81}{#1}}
\newcommand{\DocumentationTok}[1]{\textcolor[rgb]{0.56,0.35,0.01}{\textbf{\textit{#1}}}}
\newcommand{\ErrorTok}[1]{\textcolor[rgb]{0.64,0.00,0.00}{\textbf{#1}}}
\newcommand{\ExtensionTok}[1]{#1}
\newcommand{\FloatTok}[1]{\textcolor[rgb]{0.00,0.00,0.81}{#1}}
\newcommand{\FunctionTok}[1]{\textcolor[rgb]{0.00,0.00,0.00}{#1}}
\newcommand{\ImportTok}[1]{#1}
\newcommand{\InformationTok}[1]{\textcolor[rgb]{0.56,0.35,0.01}{\textbf{\textit{#1}}}}
\newcommand{\KeywordTok}[1]{\textcolor[rgb]{0.13,0.29,0.53}{\textbf{#1}}}
\newcommand{\NormalTok}[1]{#1}
\newcommand{\OperatorTok}[1]{\textcolor[rgb]{0.81,0.36,0.00}{\textbf{#1}}}
\newcommand{\OtherTok}[1]{\textcolor[rgb]{0.56,0.35,0.01}{#1}}
\newcommand{\PreprocessorTok}[1]{\textcolor[rgb]{0.56,0.35,0.01}{\textit{#1}}}
\newcommand{\RegionMarkerTok}[1]{#1}
\newcommand{\SpecialCharTok}[1]{\textcolor[rgb]{0.00,0.00,0.00}{#1}}
\newcommand{\SpecialStringTok}[1]{\textcolor[rgb]{0.31,0.60,0.02}{#1}}
\newcommand{\StringTok}[1]{\textcolor[rgb]{0.31,0.60,0.02}{#1}}
\newcommand{\VariableTok}[1]{\textcolor[rgb]{0.00,0.00,0.00}{#1}}
\newcommand{\VerbatimStringTok}[1]{\textcolor[rgb]{0.31,0.60,0.02}{#1}}
\newcommand{\WarningTok}[1]{\textcolor[rgb]{0.56,0.35,0.01}{\textbf{\textit{#1}}}}
\usepackage{graphicx}
\makeatletter
\def\maxwidth{\ifdim\Gin@nat@width>\linewidth\linewidth\else\Gin@nat@width\fi}
\def\maxheight{\ifdim\Gin@nat@height>\textheight\textheight\else\Gin@nat@height\fi}
\makeatother
% Scale images if necessary, so that they will not overflow the page
% margins by default, and it is still possible to overwrite the defaults
% using explicit options in \includegraphics[width, height, ...]{}
\setkeys{Gin}{width=\maxwidth,height=\maxheight,keepaspectratio}
% Set default figure placement to htbp
\makeatletter
\def\fps@figure{htbp}
\makeatother
\setlength{\emergencystretch}{3em} % prevent overfull lines
\providecommand{\tightlist}{%
  \setlength{\itemsep}{0pt}\setlength{\parskip}{0pt}}
\setcounter{secnumdepth}{-\maxdimen} % remove section numbering
\ifLuaTeX
  \usepackage{selnolig}  % disable illegal ligatures
\fi

\begin{document}
\maketitle

\#Run this code first

\begin{center}\rule{0.5\linewidth}{0.5pt}\end{center}

\hypertarget{if-you-dont-know-the-answer-leave-it-blank.-if-you-are-caught-cheating-you-will-be-given-minus-50-points.}{%
\subsubsection{If you don't know the answer, leave it blank. If you are
caught cheating, you will be given minus 50
points.}\label{if-you-dont-know-the-answer-leave-it-blank.-if-you-are-caught-cheating-you-will-be-given-minus-50-points.}}

\begin{center}\rule{0.5\linewidth}{0.5pt}\end{center}

Q1. Replace the author name with your name in YAML part above.

Q2. Store five values \texttt{82.0,\ 31.2,\ 98.2,\ 19.4,\ 72.6} into the
\texttt{scores} variable.

\begin{Shaded}
\begin{Highlighting}[]
\NormalTok{scores }\OtherTok{\textless{}{-}} \FunctionTok{c}\NormalTok{(}\FloatTok{82.0}\NormalTok{, }\FloatTok{31.2}\NormalTok{, }\FloatTok{98.2}\NormalTok{, }\FloatTok{19.4}\NormalTok{, }\FloatTok{72.6}\NormalTok{)}
\end{Highlighting}
\end{Shaded}

Q3. Write a code that finds the minimun value of \texttt{scores} that
you have created in Q2.

\begin{Shaded}
\begin{Highlighting}[]
\FunctionTok{min}\NormalTok{(scores)}
\end{Highlighting}
\end{Shaded}

\begin{verbatim}
## [1] 19.4
\end{verbatim}

Q4. Assign the value of 4 raised to 2 to a variable \texttt{generation}.
Then, print out the value of \texttt{generation}.

\begin{Shaded}
\begin{Highlighting}[]
\NormalTok{generation }\OtherTok{\textless{}{-}} \DecValTok{4}\SpecialCharTok{\^{}}\DecValTok{2}
\NormalTok{generation}
\end{Highlighting}
\end{Shaded}

\begin{verbatim}
## [1] 16
\end{verbatim}

Q5. Assign the value of square root 81 to a variable \texttt{nine}, and
print out \texttt{nine}.

\begin{Shaded}
\begin{Highlighting}[]
\NormalTok{nine }\OtherTok{\textless{}{-}} \FunctionTok{sqrt}\NormalTok{(}\DecValTok{81}\NormalTok{)}
\NormalTok{nine}
\end{Highlighting}
\end{Shaded}

\begin{verbatim}
## [1] 9
\end{verbatim}

Q6. Store a text \texttt{mozart} into the variable \texttt{piano}.

\begin{Shaded}
\begin{Highlighting}[]
\NormalTok{pinao }\OtherTok{\textless{}{-}} \StringTok{"mozart"}
\end{Highlighting}
\end{Shaded}

Q7. What are three components for a single plot of \texttt{ggplot2}
package?

\begin{Shaded}
\begin{Highlighting}[]
\CommentTok{\#ggplot(data,mapping,geom)}
\CommentTok{\#three components are: data, mapping, geom}
\end{Highlighting}
\end{Shaded}

Q8. A line of code that shows \texttt{presidential} data as a table

\begin{Shaded}
\begin{Highlighting}[]
\FunctionTok{View}\NormalTok{(presidential)}
\end{Highlighting}
\end{Shaded}

Q9. Create a matrix with 4 rows that contain the numbers 1 up to 12

\begin{Shaded}
\begin{Highlighting}[]
\NormalTok{data1 }\OtherTok{=} \FunctionTok{c}\NormalTok{(}\DecValTok{1}\NormalTok{,}\DecValTok{2}\NormalTok{,}\DecValTok{3}\NormalTok{,}\DecValTok{4}\NormalTok{,}\DecValTok{5}\NormalTok{,}\DecValTok{6}\NormalTok{,}\DecValTok{7}\NormalTok{,}\DecValTok{8}\NormalTok{,}\DecValTok{9}\NormalTok{,}\DecValTok{10}\NormalTok{,}\DecValTok{11}\NormalTok{,}\DecValTok{12}\NormalTok{)}
\FunctionTok{matrix}\NormalTok{(data1, }\DecValTok{4}\NormalTok{, }\DecValTok{3}\NormalTok{)}
\end{Highlighting}
\end{Shaded}

\begin{verbatim}
##      [,1] [,2] [,3]
## [1,]    1    5    9
## [2,]    2    6   10
## [3,]    3    7   11
## [4,]    4    8   12
\end{verbatim}

Q10. A line of code that assigns \texttt{displ} column as
\texttt{x-axis} and \texttt{hwy} column as \texttt{y-axis} of
\texttt{mpg} data to a variable \texttt{mpg\_plot} using
\texttt{ggplot2} package

\begin{Shaded}
\begin{Highlighting}[]
\NormalTok{mpg\_plot }\OtherTok{\textless{}{-}} \FunctionTok{ggplot}\NormalTok{(}\AttributeTok{data=}\NormalTok{mpg, }\FunctionTok{aes}\NormalTok{(}\AttributeTok{x=}\NormalTok{displ, }\AttributeTok{y=}\NormalTok{hwy))}
\NormalTok{mpg\_plot}
\end{Highlighting}
\end{Shaded}

\includegraphics{HW1_files/figure-latex/unnamed-chunk-10-1.pdf}

Q11. Two lines of code that create a scatter plot of a variable
\texttt{mpg\_plot} that you have made in Q10

\begin{Shaded}
\begin{Highlighting}[]
\NormalTok{mpg\_plot }\SpecialCharTok{+} 
  \FunctionTok{geom\_point}\NormalTok{()}
\end{Highlighting}
\end{Shaded}

\includegraphics{HW1_files/figure-latex/unnamed-chunk-11-1.pdf}

Q12. Three lines of code that create subplots (four rows) by
\texttt{class} column, using two lines of code for Q10.

\begin{Shaded}
\begin{Highlighting}[]
\NormalTok{mpg\_plot }\SpecialCharTok{+}
  \FunctionTok{geom\_col}\NormalTok{()}
\end{Highlighting}
\end{Shaded}

\includegraphics{HW1_files/figure-latex/unnamed-chunk-12-1.pdf}

Q13. A line of code that returns dimension information of
\texttt{presidential} data

\begin{Shaded}
\begin{Highlighting}[]
\FunctionTok{dim}\NormalTok{(presidential)}
\end{Highlighting}
\end{Shaded}

\begin{verbatim}
## [1] 11  4
\end{verbatim}

Q14. What are the unique values of \texttt{party} column of
\texttt{presidential} data?

\begin{Shaded}
\begin{Highlighting}[]
\FunctionTok{unique}\NormalTok{(presidential}\SpecialCharTok{$}\NormalTok{party)}
\end{Highlighting}
\end{Shaded}

\begin{verbatim}
## [1] "Republican" "Democratic"
\end{verbatim}

Q15. Two lines of code that will directly create a simple stacked bar
plot that shows the count by \texttt{class} column of \texttt{mpg} data
with filling color by \texttt{trans} column

\begin{Shaded}
\begin{Highlighting}[]
\NormalTok{bardata }\OtherTok{\textless{}{-}} \FunctionTok{table}\NormalTok{(mpg}\SpecialCharTok{$}\NormalTok{class, mpg}\SpecialCharTok{$}\NormalTok{trans)}
\FunctionTok{barplot}\NormalTok{(bardata, }\AttributeTok{col=}\FunctionTok{c}\NormalTok{(}\StringTok{"red"}\NormalTok{,}\StringTok{"orange"}\NormalTok{,}\StringTok{"yellow"}\NormalTok{,}\StringTok{"green"}\NormalTok{,}\StringTok{"blue"}\NormalTok{,}\StringTok{"purple"}\NormalTok{,}\StringTok{"hotpink"}\NormalTok{))}
\end{Highlighting}
\end{Shaded}

\includegraphics{HW1_files/figure-latex/unnamed-chunk-15-1.pdf}

Q16. A line of code that assigns \texttt{state} column as \texttt{x}
position of \texttt{midwest} data to a variable \texttt{midwest\_plot}
using \texttt{ggplot2} package

\begin{Shaded}
\begin{Highlighting}[]
\NormalTok{midwest\_plot }\OtherTok{\textless{}{-}} \FunctionTok{ggplot}\NormalTok{(midwest, }\FunctionTok{aes}\NormalTok{(state))}
\NormalTok{midwest\_plot}
\end{Highlighting}
\end{Shaded}

\includegraphics{HW1_files/figure-latex/unnamed-chunk-16-1.pdf}

Q17. Five lines of code that will return a bar plot of the
\texttt{midwest\_plot} variable with a title
\texttt{Plot\ of\ count\ by\ state}. X-axis is labeled as \texttt{state}
and y-axis as \texttt{count}.

\begin{Shaded}
\begin{Highlighting}[]
\NormalTok{midwest\_plot }\SpecialCharTok{+}
  \FunctionTok{geom\_bar}\NormalTok{()}
\end{Highlighting}
\end{Shaded}

\includegraphics{HW1_files/figure-latex/unnamed-chunk-17-1.pdf}

\begin{Shaded}
\begin{Highlighting}[]
  \FunctionTok{xlab}\NormalTok{(}\StringTok{"state"}\NormalTok{)}
\end{Highlighting}
\end{Shaded}

\begin{verbatim}
## $x
## [1] "state"
## 
## attr(,"class")
## [1] "labels"
\end{verbatim}

\begin{Shaded}
\begin{Highlighting}[]
  \FunctionTok{ylab}\NormalTok{(}\StringTok{"count"}\NormalTok{)}
\end{Highlighting}
\end{Shaded}

\begin{verbatim}
## $y
## [1] "count"
## 
## attr(,"class")
## [1] "labels"
\end{verbatim}

\begin{Shaded}
\begin{Highlighting}[]
  \FunctionTok{ggtitle}\NormalTok{(}\StringTok{"Plot of count by state"}\NormalTok{)}
\end{Highlighting}
\end{Shaded}

\begin{verbatim}
## $title
## [1] "Plot of count by state"
## 
## attr(,"class")
## [1] "labels"
\end{verbatim}

Q18. What is the name of 7th column of \texttt{diamonds} dataset?

\begin{Shaded}
\begin{Highlighting}[]
\NormalTok{diamonds[, }\FunctionTok{c}\NormalTok{(}\DecValTok{7}\NormalTok{)]}
\end{Highlighting}
\end{Shaded}

\begin{verbatim}
## # A tibble: 53,940 x 1
##    price
##    <int>
##  1   326
##  2   326
##  3   327
##  4   334
##  5   335
##  6   336
##  7   336
##  8   337
##  9   337
## 10   338
## # ... with 53,930 more rows
\end{verbatim}

Q19. How many columns and rows does \texttt{midwest} data have?

\begin{Shaded}
\begin{Highlighting}[]
\FunctionTok{nrow}\NormalTok{(midwest)}
\end{Highlighting}
\end{Shaded}

\begin{verbatim}
## [1] 437
\end{verbatim}

\begin{Shaded}
\begin{Highlighting}[]
\FunctionTok{ncol}\NormalTok{(midwest)}
\end{Highlighting}
\end{Shaded}

\begin{verbatim}
## [1] 28
\end{verbatim}

\begin{Shaded}
\begin{Highlighting}[]
\FunctionTok{dim}\NormalTok{(midwest)}
\end{Highlighting}
\end{Shaded}

\begin{verbatim}
## [1] 437  28
\end{verbatim}

\begin{Shaded}
\begin{Highlighting}[]
\CommentTok{\#midwest data has 437 rows and 28 columns}
\end{Highlighting}
\end{Shaded}

Q20. Two different commands for a quick overview of \texttt{mpg} data
that we have learned in our class

\begin{Shaded}
\begin{Highlighting}[]
\FunctionTok{View}\NormalTok{(mpg)}
\FunctionTok{summary}\NormalTok{(mpg)}
\end{Highlighting}
\end{Shaded}

\begin{verbatim}
##  manufacturer          model               displ            year     
##  Length:234         Length:234         Min.   :1.600   Min.   :1999  
##  Class :character   Class :character   1st Qu.:2.400   1st Qu.:1999  
##  Mode  :character   Mode  :character   Median :3.300   Median :2004  
##                                        Mean   :3.472   Mean   :2004  
##                                        3rd Qu.:4.600   3rd Qu.:2008  
##                                        Max.   :7.000   Max.   :2008  
##       cyl           trans               drv                 cty       
##  Min.   :4.000   Length:234         Length:234         Min.   : 9.00  
##  1st Qu.:4.000   Class :character   Class :character   1st Qu.:14.00  
##  Median :6.000   Mode  :character   Mode  :character   Median :17.00  
##  Mean   :5.889                                         Mean   :16.86  
##  3rd Qu.:8.000                                         3rd Qu.:19.00  
##  Max.   :8.000                                         Max.   :35.00  
##       hwy             fl               class          
##  Min.   :12.00   Length:234         Length:234        
##  1st Qu.:18.00   Class :character   Class :character  
##  Median :24.00   Mode  :character   Mode  :character  
##  Mean   :23.44                                        
##  3rd Qu.:27.00                                        
##  Max.   :44.00
\end{verbatim}

\begin{Shaded}
\begin{Highlighting}[]
\FunctionTok{dim}\NormalTok{(mpg)}
\end{Highlighting}
\end{Shaded}

\begin{verbatim}
## [1] 234  11
\end{verbatim}

\begin{Shaded}
\begin{Highlighting}[]
\FunctionTok{colnames}\NormalTok{(mpg)}
\end{Highlighting}
\end{Shaded}

\begin{verbatim}
##  [1] "manufacturer" "model"        "displ"        "year"         "cyl"         
##  [6] "trans"        "drv"          "cty"          "hwy"          "fl"          
## [11] "class"
\end{verbatim}

\begin{Shaded}
\begin{Highlighting}[]
\FunctionTok{unique}\NormalTok{(mpg)}
\end{Highlighting}
\end{Shaded}

\begin{verbatim}
## # A tibble: 225 x 11
##    manufacturer model      displ  year   cyl trans drv     cty   hwy fl    class
##    <chr>        <chr>      <dbl> <int> <int> <chr> <chr> <int> <int> <chr> <chr>
##  1 audi         a4           1.8  1999     4 auto~ f        18    29 p     comp~
##  2 audi         a4           1.8  1999     4 manu~ f        21    29 p     comp~
##  3 audi         a4           2    2008     4 manu~ f        20    31 p     comp~
##  4 audi         a4           2    2008     4 auto~ f        21    30 p     comp~
##  5 audi         a4           2.8  1999     6 auto~ f        16    26 p     comp~
##  6 audi         a4           2.8  1999     6 manu~ f        18    26 p     comp~
##  7 audi         a4           3.1  2008     6 auto~ f        18    27 p     comp~
##  8 audi         a4 quattro   1.8  1999     4 manu~ 4        18    26 p     comp~
##  9 audi         a4 quattro   1.8  1999     4 auto~ 4        16    25 p     comp~
## 10 audi         a4 quattro   2    2008     4 manu~ 4        20    28 p     comp~
## # ... with 215 more rows
\end{verbatim}

\hypertarget{end-of-document}{%
\subsubsection{\texorpdfstring{\emph{End of
document}}{End of document}}\label{end-of-document}}

\end{document}
